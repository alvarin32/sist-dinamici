\documentclass[a4paper,12pt]{article}
\usepackage[utf8]{inputenc}
\usepackage{amssymb, amsmath}
\newcommand{\ubar}{\underbar}
\usepackage{fullpage}

\begin{document}
\begin{titlepage}
\title{Dispensa per il primo compitino di Sistemi Dinamici}
\author{Studenti vari}
\maketitle
\end{titlepage}
\begin{titlepage}
\tableofcontents
\end{titlepage}

\section{Preambolo}
\paragraph{Unicità del problema di Cauchy}
$$\dot{x}\ddot{x}$$

\section{Equazioni del moto}

Un sistema dinamico in generale potrà essere descritto da una forma del tipo:
$$\bigg\{ \begin{array}{l}
\dot{x} = f(x, y)\\
\dot{y} = g(x, y)\\
\end{array}$$

\subsection{Diagrammi di fase}
\paragraph{Punti di equilibrio} Analiticamente:
$$f(x, y) = 0$$ $$g(x, y) = 0$$
Sono loro attorno ai quali possono avvenire le orbite (??).
\paragraph{Orbite periodiche}
Analiticamente:
$$ ( x(t+\tau), y(t+\tau) ) = (x(t), y(t))$$
Sono curve chiuse sulla quale il campo vettoriale non si annulla.
Per il teorema di esistenza e unicità non possono esservi allora orbite che la intersechino.
\paragraph{Orbite asintotiche a punti di equilibrio}
``Salvano" l'unicità della soluzione.
\paragraph{Orbite aperte non asintotiche}
Il punto che descrive lo stato del sistema le percore senza intersecare altre orbite e senza essere stato in un punto già visto.
\paragraph{Costante del moto}
È una funzione $\Phi$ tale che:
$$\Phi(x(t), y(t)) = \Phi(x_0, y_0)\ \ \forall t$$

NOTA: se $\Phi \in \mathbb{C}^1$, allora $\Phi$ è una costante del moto per il sistema
$$\bigg\{ \begin{array}{l}
\dot{x} = f(x, y)\\
\dot{y} = g(x, y)\\
\end{array}$$
se e solo se
$$f\left(\dfrac{\partial \Phi}{\partial x}\right) + g\left(\dfrac{\partial \Phi}{\partial y}\right) = 0$$
Questa proposizione serve perché a questo punto sappiamo che per schematizzare il moto basta tracciare le curve di livello identificandone eventuali singolarità. Ogni curva di livello è un insieme di orbite.

\subparagraph{Esempio - Equazioni di Newton}
$$\Biggl\{ \begin{array}{l}
\dot{x} = v\\
\dot{y} = \dfrac{F(x)}{m}\\
\end{array}$$
Flusso (???).
Tutti i punti di equilibrio sull'asse $x$ hanno $v=0$.
Punti dell'asse x non di equilibio hanno orbite a tangente verticale.

\paragraph{Sistemi conservativi}
Un sistema è conservativo se $F(x)$ ammette primitiva $U(x)$ tale che $$F(x) = -\dfrac{\partial U}{\partial x}$$ da cui $$m\ddot{x} + U'(x) = 0 \ \ \forall x$$
Per i sistemi conservativi descritti dalle equazioni di Newton l'energia è una costante del moto:
$$ E = \dfrac{1}{2} mv^2+U(x)$$
Fissata una certa energia $E_0$ è possibile il moto solo per $E\geq U(x) = E_0$ ovvero $$E-U(x)=E-E_0 = T \geq 0$$
I casi possibili sono:
\begin{itemize}
	\item $\exists x_0: \forall x < x_0,\ U(x) < E_0$ oltre che $\forall x > x_0,\ U(x) > E_0$: allora il punto proviene dall'infinito e ci ritorna; si ha v=0 quando $U(\bar{x}) = E_0$ ma $\dot{v}(\bar{x})$ non è zero, quindi l'accelerazione non è nulla
	\item Intervallo chiuso se $U(x) \leq E_0 \Rightarrow x \in (a, b) \subset \mathbb{R}$ allora si ha un'orbita periodica chiusa con estremi in cui si inverte il moto.
	\item Nota: se $a$ o $b$ sono dei punti di massimo relativi di $U(x)$, allora si hanno dei punti singolari: vedi la sezione sulle separatrici subito sotto.
\end{itemize}
\paragraph{Buche di potenziale}
Sono i punti in cui $U(x)$ ha un minimo
\paragraph{Separatrici}
In presenza di massimi di $U(x)$, il diagramma di fase è simmetrico rispetto a $x$: $E$ dipende quadraticamente da $v$, e quindi $E(x, v) = E(x, -v)$.
In particolare, ci saranno due rette che separeranno il diagramma di coefficiente angolare:
$$y'(x_0) = \pm\sqrt{-U''(x_0)}$$

\subsection{Osservazioni sui diagrammi di fase}
Se $U(x) \to E_0$ e $U$ allora $y \to 0$, e se $E_0$ non è un massimo relativo di $U(x)$, si ha una tangente verticale sul diagramma di fase.


\subsection{Moti periodici}
\paragraph{Periodo di oscillazione}
Il periodo di oscillazione (andata e ritorno) tra $x_1$ e $x_2$\footnote{dove $x_1$ e $x_2$ sono le radici di $U(x) = E $ che delimitano l'orbita chiusa! }, è dato dalla relazione
$$ T(E) = \sqrt{2}\int_{x_1(E)}^{x_2(E)} \dfrac{dx}{\sqrt{E-U(x)}}$$
\subparagraph{Stima del periodo}
A volte è difficile determinare il periodo di oscillazione attraverso il calcolo dell'integrale sopra riportato.
In questi casi, si può offrire una stima del periodo calcolando esplicitamente le energie potenziali di $x_1$ e $x_2$. In altri termini:

$$ \sqrt{2}\int_{x_1(E)}^{x_2(E)} \dfrac{dx}{\sqrt{E-U_1(x)}} \leq T \leq \sqrt{2}\int_{x_1(E)}^{x_2(E)} \dfrac{dx}{\sqrt{E-U_2(x)}} $$

\paragraph{Oscillatore armonico}
Ha un'energia pari a
$$ E = \dfrac{1}{2} m \dot{\bar{x}}^2 + \dfrac{U''(\bar{x})}{2} (x-\bar{x})$$


\paragraph{Punti di equilibrio}
Un punto di equilibrio è un punto dove, $\forall t$
$$\bigg\{ \begin{array}{l}
x(t) = x_0\\
y(t) = 0\\
\end{array}$$
\subparagraph{Equilibrio stabile}
Un punto di equilibrio $x_0$ si dice stabile, se $\forall U(x_0) \exists V(x_0) \subset U $  tale che per ogni scostamento $x_0 + h \in V$ il punto rimane in $U,\ \forall t$.
\subparagraph{Equilibrio instabile}
Un punto di equilibrio è instabile se non è stabile.

\subsection{Come risolvere un esercizio}
Facciamo innanzi tutto qualche considerazione di base. Per prima cosa, ricordiamo che:
$$y = \pm \sqrt{2(E_0-U(x))}$$
    (ovviamente il grafico risulta simmetrico rispetto all'asse x).
E quando $y(x)\to 0$ si ha che $y'(x) \to \infty$, dunque, fatta eccezione per alcuni punti singolari, la tangente al grafico è verticale.

A questo punto si comincia con l'esercizio di disegno vero e proprio:
\begin{itemize}
 \item Riconosci i punti critici dell'energia: in generale massimi, minimi di $U(x)$.
 \item Studia i limiti per $x\to\pm\infty$
 \begin{itemize}
  \item Ricorda che la distanza tra due orbite a differenti livelli energetici tende a $0$ per $x\to\infty$ {?? se U(x) avesse asintoto orizzontale ? } 
  \item Si hanno asintoti obliqui se $U(x)\asymp x^2$ (per $x\to\infty$)
 \end{itemize}

 \item Orbite chiuse:
 \begin{itemize}
  \item ...
  \item Posso calcolare periodo e pulsazione di piccole oscillazioni attorno ad un punto di equilibrio stabile ($\bar{x}$ minimo per $U(x)$) : $$\omega = \sqrt{U''(\bar{x})\over m};\ \ \ T = 2\pi\sqrt{m \over U''(\bar{x})}$$
 \end{itemize}
 \item Orbite aperte:
 \begin{itemize}
  \item ...
 \end{itemize}
 \item Casi particolari:
 \begin{itemize}
   \item Intorno ai punti di massimo ci sarà un'orbita in cui il grafico incontra l'asse $x$ (supponiamo, ad esempio, nel punto $(x_0, 0)$), arrivando a tangere le rette di equazione $$y = \pm\sqrt{-\dfrac{U''(x_0)}{m}}(x-x_0)$$
 \end{itemize}
\end{itemize}

\paragraph{Punti di equilibrio}
Si possono utilizzare diversi metodi per il calcolo dei punti di equilbirio. Il più semplice è qui illustrato.
Per prima cosa linearizzo il sistema:

$$\begin{bmatrix}\dot{\xi} \\ \dot{\eta} \end{bmatrix} 
= \begin{bmatrix} \partial_xf(x_0,y_0) & \partial_yf(x_0,y_0) \\ \partial_xg(x_0,y_0) & \partial_yg(x_0,y_0) \end{bmatrix} \begin{bmatrix}\xi \\ \eta \end{bmatrix}$$

A questo punto posso calcolare gli autovalori di questo sistema. Se gli autovalori di questo sistema hanno parte reale positiva, il sistema originario è instabile. Viceversa, se hanno parte reale negativa, allora il sistema è stabile in quel punto.

\paragraph{Calcolo del tempo di raggiungimento di una posizione arbitraria}
Bisogna usare la definizione (SCRIVI).
In particolare abbiamo i casi.
\subparagraph{Punti non stazionari}
Sia (ad esempio): $U'(x_0) = A > 0$, il punto non è dunque critico.
$$t(x_0) = \int_x^{x_0} \dfrac{d\tilde{x}}{\sqrt{2}A\sqrt{|x-x_0|}} < +\infty$$
dunque, finché non ho punti stazionari, posso raggiungerli in tempi finiti.
\subparagraph{Orbite chiuse}
Se E è un'orbita chiusa (con estremi $x_1$ e $x_2$)
$$ T(E) = \sqrt{2}\int_{x_1(E)}^{x_2(E)} \dfrac{dx}{\sqrt{E-U(x)}}$$
\subparagraph{Punti stazionari}
Sia $p$ un massimo.
Allora
$$t(p) = \lim_{x\to p} \int_{x_0}^x \dfrac{d\tilde{x}}{\sqrt{E_2 - U(\tilde{x})}} = +\infty$$
Dunque, se ho energia $E = U(p)$ non raggiungerò $p$ in un tempo finito.

\paragraph{Stima del periodo di oscillazione}


\paragraph{Sistemi perturbati}
Si cerca una funzione di Lyapunov (v. sotto) per il sistema non perturbato (ad esempio: l'energia) e si guarda se va bene anche per quello perturbato. Si può anche scegliere $L$ tale che $$\dfrac{\partial L}{\partial x}\dot{x} + \dfrac{\partial L}{\partial y}\dot{y} = 0 $$ e vedere se va bene.

\subsection{Ricorda}

\paragraph{Funzioni di Lyapunov}
Siano
$$\bigg\{ \begin{array}{l}
\dot{x} = f(x,y)\\
\dot{y} = g(x,y)\\
\end{array}$$
E sia $(x_0, y_0)$ t.c.
$$\bigg\{ \begin{array}{l}
f(x_0,y_0) = 0\\
g(x_0,y_0) = 0\\
\end{array}$$
Allora se
$$ \dfrac{dL}{dt} = \dfrac{\partial L}{\partial x}\dot{x} + \dfrac{\partial L}{\partial y}\dot{y} \leq 0 $$
(ovvero $L(x, y)$ ha minimo stretto in $(x_0, y_0)$), si ha che $(x_0,y_0)$ è un punto di equilibrio stabile.

\paragraph{Teoremi utili}
$$T(E) = \dfrac{d}{dt}S(E)$$
dove $S(E)$ è l'area dell'orbita chiusa periodica.

Poi, se $E_0$ è l'energia corrispondente a $U(x_m)$ allora $$\lim_{E\to E_0} T(E) = \dfrac{2\pi}{\sqrt{U''(x_m)}}$$


\section{Meccanica Lagrangiana}
\subsection{Definizioni principali}
La funzione di Lagrange è la funzione:
$$ L = T - U $$
e soddisfa la relazione
$${d \over dt}\dfrac{\partial L}{\partial\dot{q}_i} = \dfrac{\partial L}{\partial q_i}$$
\subsection{Come risolvere un esercizio}

\subsection{Ricorda}
\paragraph{Teorema di Koenig}

\paragraph{Teorema di Huygens-Steiner}
Ricorda: non puoi usarlo se: (nota di lorenzoni)

\section{Marco}	
\begin{itemize}


\item Se scriviamo $T=T_2+T_1+T_0$ e quindi $E=T_2-T_0+V$ \\
POTENZIALE CENTRALE NEL PIANO \\
Quando sappiamo che il momento angolare è costante otteniamo:
\begin{displaymath}
m\ddot{r}=\frac{l^2}{mr^3}-\frac{dv}{dr}
\end{displaymath}

DESCRIVE L'EVOLUZIONE DEL RAGGIO.\\
E' l'equazione di un punto materiale in movimento sotto l'azione del potenziale efficace $V^*(r)=V(r)+\frac{l^2}{mr^2}$ dove l'ultimo termine è il potenziale centrifugo
\begin{displaymath}
E=\frac{1}{2}mr^2 + V^*(r)
\end{displaymath}
DIAGRAMMA DI FASE DI UN POTENZIALE RADIALE
\item $E_{min}=V^*(F)$ corrisponde a $r(t)=\overline{r}$ costante
\item $E<0$ STATI LEGATI: corrisponde a un moto confinato a distanza finita dal centro
\item $E=0,\ \dot{r}\to0$, per $r\to\infty$
\item $E>0$ STATI DI SCATTERING: per $r\to\infty$, $\dot{r}\to\sqrt{\frac{2E}{m}}$, proviene dall'infinito e ci ritorna
\end{itemize}
\newtheorem{prop}{Proposizione}[section]

\begin{prop}
In qualsiasi campo di fase centrale verificati i seguenti fatti:
\begin{itemize} 
\item $\vec{L}$ è costante del moto
\item La velocità areolare è costante, ossia $\frac{dA}{dt}=cost$

\end{itemize}
 

\end{prop}
\paragraph{Osservazione} Di conseguenza al teorema si ha che $mp^2\dot{\theta}$ è un integrale primo
\paragraph{Orbite degli stati legati}
L'orbita è contenuta in una corona circolare compresa tra i raggi $\rho_{min}$ e $\rho_{max}$ e al crescere di $\theta$ oscilla tra i due estremi. Si avrà un'orbita a \textbf{ rosetta}. E' periodica se e solo se $\frac{\Delta\theta}{2\pi}\in\mathbb{Q}$
\paragraph{Energia cinetica in coordinate sferiche}
\begin{displaymath}
T=\frac{1}{2}ml^2(\dot{\theta^2}+\sin^2\theta \cdot \dot{\phi^2})
\end{displaymath}
$P_{\phi}$ si conserva se e solo se $\vec{L_z}$ si conserva, ove $\vec{L_z}=mr^2\sin^2\theta\cdot\dot{\phi}$
\paragraph{Integrazione per quadrature}
$E=T+U$ sostituisco in coordinate $\dot{\phi^{\alpha}}$ e trovo $E=\frac{1}{2}\dot{\theta^2}U^{\alpha}(\theta)$

\section{Altro}
\paragraph{Regola di Cramer}
\paragraph{Autovalori e autovettori}

\section{Sistemi lineari}
\paragraph{Linearizzazione di un sistema}
E' dato un sistema lineare ed un suo punto di equilibrio $(x_0,y_0)$ (cioè tale che $f(x_0,y_0)=0=g(x_0,y_0)$ (*))
$$\bigg\{ \begin{array}{l}
\dot{x} = f(x,y)\\
\dot{y} = g(x,y)\\
\end{array}$$
Cambio di variabili:
$$\bigg\{ \begin{array}{l}
\xi = x-x_0\\
\eta = y-y_o\\
\end{array}$$
Riscriviamo il sistema con le nuove variabili e scriviamo lo sviluppo di Taylor di f e g fino al I ordine, ricordando (*).
$$\bigg\{ \begin{array}{l}
\dot{\xi} = \partial_xf(x_0,y_0)\xi+\partial_yf(x_0,y_0)\eta + o(\sqrt{\xi^2+\eta^2})\\
\dot{\eta} = \partial_xg(x_0,y_0)\xi+\partial_yg(x_0,y_0)\eta + o(\sqrt{\xi^2+\eta^2})\\
\end{array}$$ 
Dal precedente otteniamo il sistema lineare: 
$$\begin{bmatrix}\dot{\xi} \\ \dot{\eta} \end{bmatrix} 
= \begin{bmatrix} \partial_xf(x_0,y_0) & \partial_yf(x_0,y_0) \\ \partial_xg(x_0,y_0) & \partial_yg(x_0,y_0) \end{bmatrix} \begin{bmatrix}\xi \\ \eta \end{bmatrix}
$$
\paragraph{Studio di un generico sistema lineare}
Un generico sistema lineare, scritto in forma matriciale:
$$\begin{bmatrix}\dot{x} \\ \dot{y} \end{bmatrix} 
=  \underbrace{\begin{bmatrix} a & b \\ c & d \end{bmatrix}}_\text{A}  \begin{bmatrix}x \\ y \end{bmatrix}
$$
Polinomio caratteristico $P_A(\lambda) = \lambda^2-tr(A)\lambda+det(A)$
\section{Teoremi generali}
\paragraph{}
$$
\vec{P} = m\vec{v}$$
$$
\vec{L_Q} = (x-x_Q)\wedge P
$$



\end{document}


